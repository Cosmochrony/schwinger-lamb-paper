\section{The Lamb Shift as Projective Spectral Noise}
  \label{sec:lamb-shift}

  The Lamb shift constitutes one of the most precise and conceptually significant
  tests of quantum electrodynamics.
  Experimentally observed as a lifting of the degeneracy between the
  $2s_{1/2}$ and $2p_{1/2}$ levels in hydrogen, it played a decisive role in the
  development of renormalized QED~\cite{Lamb1947}.
  In the standard interpretation, this shift is attributed to radiative corrections,
  notably electron self-energy and vacuum polarization effects.

  From the projective perspective introduced in the previous section, the Lamb shift
  admits an alternative interpretation.
  Rather than signaling the interaction of an electron with fluctuating vacuum fields,
  the observed energy correction reflects a limitation in the resolvability of highly
  localized interactions under the non-injective projection
  $\Pi : \Omega \rightarrow \mathcal{O}$.
  The Lamb shift thus appears as a manifestation of projective spectral noise.

  A key empirical feature of the Lamb shift is its pronounced dependence on the spatial
  localization of the electronic wavefunction.
  States with nonzero probability density at the origin, such as $s$-states, are
  systematically affected, while states with vanishing density at the nucleus, such as
  $p$-states, exhibit substantially reduced corrections.
  This sensitivity to short-distance structure is traditionally interpreted as
  evidence for ultraviolet vacuum fluctuations.

  In Cosmochrony, the same sensitivity arises naturally from the structure of the
  projection itself.
  Highly localized states probe regions of elevated relational density, where distinct
  configurations of the substrate~$\chi$ are compressed into indistinguishable
  effective observables.
  The projection $\Pi$ therefore fails to preserve fine-grained spectral distinctions,
  introducing an irreducible uncertainty in the effective energy levels.

  This uncertainty can be characterized as a form of spectral noise associated with the
  local relational Laplacian governing the effective atomic dynamics.
  While the detailed spectrum of this Laplacian remains inaccessible at the level of
  effective spacetime observables, its coarse-grained influence manifests as systematic
  energy shifts.
  The lifting of the $s$--$p$ degeneracy thus reflects the differential sensitivity of
  atomic states to unresolved relational modes, rather than a dynamical self-interaction
  of the electron.

  Importantly, this reinterpretation preserves the quantitative success of QED.
  The standard radiative corrections may be viewed as an effective encoding of the same
  structural limitation, translated into the language of field fluctuations and
  renormalization.
  From the projective standpoint, renormalization does not eliminate divergences of a
  physical vacuum, but compensates for the inability of the effective description to
  faithfully represent arbitrarily fine relational structure.

  The Lamb shift therefore acquires a clear geometric and relational meaning.
  It signals the breakdown of perfect injectivity of the projection in regimes of
  extreme localization, where the resolution scale approaches the fundamental bound
  imposed by the structure of $\Pi$.
  In this sense, the shift is neither accidental nor specific to electromagnetic
  interactions, but represents a generic feature of emergent dynamics in a
  pre-geometric framework.

  This projective interpretation also suggests a natural hierarchy of corrections.
  As atomic number increases and electronic states probe progressively smaller length
  scales, deviations from idealized degeneracies are expected to scale with the local
  degree of projective compression.
  Precision spectroscopy thus provides a direct observational window into the
  resolution limits of the effective spacetime description.

  In the next section, we show that an analogous saturation mechanism operates in the
  strong-field regime.
  There, the limitation is not spectral resolution but the transport capacity of
  relational flux, leading to the non-perturbative instability known as the Schwinger
  effect.
