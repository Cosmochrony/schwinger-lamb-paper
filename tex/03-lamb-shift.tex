\section{The Lamb Shift as Projective Spectral Noise}
  \label{sec:lamb-shift}

  The Lamb shift constitutes one of the most precise and conceptually significant
  tests of quantum electrodynamics.
  Experimentally observed as a lifting of the degeneracy between the
  $2s_{1/2}$ and $2p_{1/2}$ levels in hydrogen, it played a decisive role in the
  development of renormalized QED~\cite{Lamb1947}.
  In the standard interpretation, this shift is attributed to radiative corrections,
  notably electron self-energy and vacuum polarization effects.

  From the projective perspective introduced in Section~\ref{sec:projective-vacuum},
  the Lamb shift admits a different interpretation.
  Rather than signaling an interaction with fluctuating vacuum degrees of freedom,
  the observed energy correction reflects a limitation in the spectral resolvability
  of highly localized bound states under the non-injective projection
  $\Pi : \Omega \rightarrow \mathcal{O}$.
  The Lamb shift thus appears as a manifestation of projective spectral noise.

  Atomic bound states correspond, in this framework, to admissible localized relaxation
  modes of the relational substrate.
  While the linear effective description predicts degeneracies between states of equal
  principal quantum number, this degeneracy is lifted once non-linear saturation and
  projectability constraints are taken into account.
  The lifting is not uniform across orbital configurations, but depends sensitively on
  the degree to which a given state probes the inner structure of the projection.

  States with nonzero probability density at the origin, such as $s$-states
  ($\ell=0$), probe regions of elevated relational density, where coupling between
  localized modes and the global relaxation flow is strongest.
  In these regions, Born--Infeld-type saturation effects and spectral frustration are
  maximal.
  By contrast, $p$-states ($\ell=1$), whose wavefunctions vanish at the nucleus, remain
  less sensitive to these inner-core constraints.
  The observed $s$--$p$ splitting therefore reflects a differential sensitivity to
  unresolved relational structure rather than a dynamical self-interaction of the
  electron.

  At the effective level, this mechanism induces a finite upward shift of the $s$-state
  energy.
  Dimensional considerations suggest a characteristic scale of the form
  \begin{equation}
    \Delta E_{\mathrm{Lamb}}
    \sim \kappa\, \alpha^5 m_e c^2 ,
  \end{equation}
  where $\alpha$ denotes the invariant ratio between projective resolution and
  relational flux capacity, and $\kappa$ is a numerical factor of order unity encoding
  details of the local spectral structure.
  This estimate yields an energy shift of order $10^{-5}\,\mathrm{eV}$, consistent with
  the observed magnitude of the Lamb splitting.
  Crucially, the correction is intrinsically finite, reflecting the bounded nature of
  the underlying relational dynamics.

  From this viewpoint, renormalization in QED can be reinterpreted as an effective
  procedure compensating for the coarse-grained representation of unresolved spectral
  structure.
  The quantitative success of standard radiative corrections is preserved, but their
  ontological interpretation is revised.
  The Lamb shift does not arise from vacuum fluctuations, but from spectral frustration
  induced by the finite resolution and saturation properties of the projection.

  This spectral-probe logic extends naturally to other fine and hyperfine corrections,
  which similarly depend on the proximity of localized electronic states to regions of
  maximal projective compression.
  Precision spectroscopy thus provides a direct observational window into the spectral
  limits of emergent spacetime dynamics.
