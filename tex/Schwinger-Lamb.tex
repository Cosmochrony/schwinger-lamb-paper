\documentclass[pdflatex,sn-mathphys-num]{sn-jnl} % Math and Physical Sciences Numbered Reference Style
\usepackage{amsmath,amssymb,amsfonts}
\usepackage{graphicx}
\usepackage{hyperref}
\usepackage{physics}
\usepackage{tikz}
\usepackage{amsopn}
\usepackage{amstext}
\usepackage{microtype}
\usepackage{xparse}
\usepackage{algorithm}
\usepackage{algorithmicx}
\usepackage{algpseudocode}
% Boxes
\usepackage[most]{tcolorbox}
% Plots (pgfplots provides axis/addplot)
\usepackage{pgfplots}
\usetikzlibrary{intersections}
\usepgfplotslibrary{fillbetween}
\pgfplotsset{compat=1.18}
\NewDocumentCommand{\subsubsubsection}{s m}{%
  \IfBooleanTF{#1}{\paragraph*{#2}}{\paragraph{#2}}%
}
\usetikzlibrary
{arrows.meta, calc, fit, decorations.pathmorphing, decorations.pathreplacing, shapes.geometric, shadings, positioning,
  decorations.markings, patterns}
\theoremstyle{remark}
\newtheorem*{remark}{Remark}

\title{The Fine-Structure Constant as a Projective Invariant:
Lamb Shift and Schwinger Effect in a Pre-Geometric Framework}
\author*[1]{\fnm{Jérôme} \sur{Beau}}\email{jerome.beau@cosmochrony.org}
\affil*[1]{\orgname{Independent Researcher}, \country{France}}
\date{}

\begin{document}

  \abstract{
    The Lamb shift and the Schwinger effect are among the most precise and conceptually
    challenging predictions of quantum electrodynamics.
    They are conventionally interpreted as consequences of vacuum fluctuations,
    radiative corrections, and non-perturbative instabilities of a dynamical quantum
    vacuum.
    While this interpretation achieves remarkable quantitative success, it leaves
    open fundamental questions regarding the ontological status of the vacuum and the
    origin of renormalization procedures.

    In this work, we propose a unified reinterpretation of these phenomena within a
    relational pre-geometric description, in which all effective physical observables
    emerge from a non-injective projection of an underlying relational substrate.
    From this perspective, vacuum-related effects do not arise from physical excitations
    of an underlying field-theoretic vacuum, but from intrinsic limitations of the
    projection mapping relational configurations to effective spacetime descriptions.

    We show that the Lamb shift can be understood as projective spectral noise, reflecting
    the finite resolvability of highly localized interactions and the coarse-grained
    influence of unresolved relational modes.
    Similarly, the Schwinger effect is reinterpreted as a saturation phenomenon associated
    with the bounded capacity of the projection to transport relational flux under extreme
    field conditions, leading to a breakdown of injectivity and the emergence of new
    effective degrees of freedom.

    A central result of this analysis is a unified structural interpretation of the
    fine-structure constant.
    Rather than appearing as an unexplained fundamental parameter, it emerges as an
    invariant ratio between projective resolution and relational flux capacity, governing
    both atomic-scale spectral corrections and strong-field instabilities.

    This reinterpretation preserves the quantitative predictions of quantum
    electrodynamics while providing a coherent ontological account of its most subtle
    effects.
    It demonstrates that precision QED phenomena can be consistently embedded within a
    pre-geometric relational description, supporting the view that quantum field theories
    operate as effective descriptions of deeper projective constraints rather than as
    fundamental theories of the vacuum.
  }

  \maketitle

  \keywords{
    quantum electrodynamics;
    Lamb shift;
    Schwinger effect;
    vacuum interpretation;
    non-perturbative phenomena;
    emergent spacetime;
    pre-geometric frameworks
  }

  \section{Introduction and Motivation}
  \label{sec:introduction}

  Quantum Electrodynamics (QED) stands as one of the most accurate physical theories
  ever constructed, achieving extraordinary agreement between theoretical predictions
  and experimental measurements across a wide range of phenomena~\cite{Weinberg1995}.
  Among its most emblematic successes are the Lamb shift in hydrogenic atoms and the
  non-perturbative prediction of electron-positron pair production in ultra-strong
  electric fields, known as the Schwinger effect~\cite{Lamb1947,Schwinger1951}.

  Despite their empirical precision, these phenomena raise persistent conceptual
  questions regarding the ontological status of the quantum vacuum.
  In the standard formulation, both effects are attributed to vacuum fluctuations
  and radiative corrections arising from the interaction of charged particles with
  quantized electromagnetic fields.
  The vacuum is thus treated as a dynamical entity, endowed with structure, energy,
  and virtual excitations, whose effects must be regularized and renormalized to
  yield finite physical predictions~\cite{PeskinSchroeder1995}.

  Within the Cosmochrony framework, the quantum vacuum is not regarded as a fundamental
  physical medium.
  Instead, all effective physical observables arise from a non-injective projection
  $\Pi$ of a single pre-geometric relational substrate~$\chi$ onto an effective
  spacetime description.
  As developed in previous works, the non-injectivity of this projection implies that
  distinct relational configurations may correspond to identical effective observables,
  leading to intrinsic limits in localization, resolution, and transport~\cite{Beau2026a}.

  From this perspective, phenomena traditionally attributed to vacuum fluctuations
  may be reinterpreted as manifestations of projective limitations rather than
  dynamical excitations of an underlying field.
  In regimes of high informational density or extreme field intensity, the projection
  $\Pi$ ceases to be effectively injective, and observable corrections emerge as
  structural effects associated with bounded resolution and saturation of relational
  flux.

  The present work applies this projective interpretation to two cornerstone effects
  of QED: the Lamb shift and the Schwinger pair production mechanism.
  We argue that the Lamb shift can be understood as a form of projective spectral noise,
  arising from the finite resolvability of interactions localized at atomic scales.
  In this view, the energy level shifts of $s$-states relative to $p$-states do not
  reflect electron self-energy in a fluctuating vacuum, but the sensitivity of highly
  localized states to the spectral structure of the underlying relational Laplacian.

  Similarly, the Schwinger effect is reinterpreted as a saturation phenomenon associated
  with a bounded capacity of the relational substrate to transport effective
  electromagnetic flux.
  When this capacity is exceeded, the projection becomes unstable, and the system
  relaxes through a local re-stratification of the substrate, manifesting as the
  creation of particle-antiparticle pairs.
  Pair production is therefore not viewed as extraction from the vacuum, but as an
  inevitable consequence of non-injective projection under extreme field conditions.

  A central outcome of this unified interpretation is a revised understanding of the
  fine-structure constant~$\alpha$.
  Rather than appearing as an unexplained fundamental parameter, $\alpha$ emerges as
  an invariant ratio between projective resolution and relational flux capacity,
  governing both the magnitude of atomic-level corrections and the threshold behavior
  of strong-field phenomena.

  The goal of this paper is not to challenge the quantitative success of QED, but to
  provide an alternative ontological interpretation of its most subtle predictions.
  By reframing the Lamb shift and the Schwinger effect as projective and saturation
  phenomena, respectively, this work aims to demonstrate that precision quantum
  electrodynamics can be consistently embedded within a pre-geometric relational
  framework.
  This embedding offers a coherent account of vacuum-related effects without invoking
  a physically populated vacuum, and strengthens the claim that Cosmochrony provides a
  unified effective description spanning gravitation, quantum non-locality, and
  electromagnetic interactions.

  \section{Projective Origin of Vacuum Effects}
  \label{sec:projective-vacuum}

  In standard quantum field theory, vacuum-related effects are commonly interpreted
  as arising from fluctuations of quantized fields around a lowest-energy state.
  These fluctuations are treated as physical, albeit virtual, entities whose
  observable consequences emerge through radiative corrections and loop processes.
  While this interpretation has proven extraordinarily successful at the
  computational level, it implicitly attributes a rich dynamical structure to the
  vacuum itself.

  Within the Cosmochrony framework, no such physically populated vacuum is postulated.
  Instead, all effective physical phenomena originate from a single pre-geometric
  relational substrate~$\chi$, whose configurations are mapped to effective
  spacetime observables through a projection
  $\Pi : \Omega \rightarrow \mathcal{O}$.
  As established in earlier works, this projection is generically non-injective:
  distinct relational configurations may correspond to identical effective
  descriptions~\cite{Beau2026a}.

  Non-injectivity of the projection has direct and unavoidable physical consequences.
  In particular, it implies that the effective description cannot resolve arbitrarily
  fine relational distinctions.
  This limitation introduces a fundamental bound on localization, timing, and
  interaction resolution, independent of any dynamical fluctuations.
  Observable deviations from idealized point-like behavior therefore arise as
  structural features of the projection itself.

  We refer to this limitation as \emph{projective resolution}.
  At sufficiently low energies and moderate field intensities, the projection
  $\Pi$ behaves as effectively injective, and standard quantum and classical
  descriptions remain accurate.
  However, in regimes characterized by extreme localization or high relational
  flux, the projection saturates.
  Beyond this regime, additional relational information cannot be faithfully
  transported into the effective spacetime description.

  The saturation of projective resolution manifests observationally as corrections
  to otherwise degenerate or symmetric configurations.
  Crucially, these corrections do not signal the presence of additional degrees of
  freedom, but rather the breakdown of perfect correspondence between relational
  configurations and effective observables.
  In this sense, so-called vacuum effects are reinterpreted as \emph{projection
artifacts}, encoding the finite capacity and resolution of the mapping from
  $\chi$ to spacetime.

  This perspective provides a unified reinterpretation of phenomena traditionally
  treated as conceptually distinct.
  Corrections attributed to vacuum polarization, self-energy, or zero-point
  fluctuations are recast as manifestations of bounded projectability.
  They reflect the fact that effective fields and particles are not fundamental
  entities, but emergent descriptions constrained by the structural properties of
  the underlying relational substrate.

  In the following sections, this projective interpretation is applied to two
  paradigmatic quantum electrodynamical phenomena.
  The Lamb shift is shown to arise from the sensitivity of highly localized atomic
  states to projective spectral noise.
  The Schwinger effect is interpreted as a saturation-induced instability, in which
  excess relational flux forces a local reconfiguration of the substrate.
  Together, these effects illustrate how precision QED phenomena emerge naturally
  from the same projective principles governing gravitation and cosmology within
  the Cosmochrony framework.

  \section{The Lamb Shift as Projective Spectral Noise}
  \label{sec:lamb-shift}

  The Lamb shift constitutes one of the most precise and conceptually significant
  tests of quantum electrodynamics.
  Experimentally observed as a lifting of the degeneracy between the
  $2s_{1/2}$ and $2p_{1/2}$ levels in hydrogen, it played a decisive role in the
  development of renormalized QED~\cite{Lamb1947}.
  In the standard interpretation, this shift is attributed to radiative corrections,
  notably electron self-energy and vacuum polarization effects.

  From the projective perspective introduced in Section~\ref{sec:projective-vacuum},
  the Lamb shift admits a different interpretation.
  Rather than signaling an interaction with fluctuating vacuum degrees of freedom,
  the observed energy correction reflects a limitation in the spectral resolvability
  of highly localized bound states under the non-injective projection
  $\Pi : \Omega \rightarrow \mathcal{O}$.
  The Lamb shift thus appears as a manifestation of projective spectral noise.

  Atomic bound states correspond, in this framework, to admissible localized relaxation
  modes of the relational substrate.
  While the linear effective description predicts degeneracies between states of equal
  principal quantum number, this degeneracy is lifted once non-linear saturation and
  projectability constraints are taken into account.
  The lifting is not uniform across orbital configurations, but depends sensitively on
  the degree to which a given state probes the inner structure of the projection.

  States with nonzero probability density at the origin, such as $s$-states
  ($\ell=0$), probe regions of elevated relational density, where coupling between
  localized modes and the global relaxation flow is strongest.
  In these regions, Born--Infeld-type saturation effects and spectral frustration are
  maximal.
  By contrast, $p$-states ($\ell=1$), whose wavefunctions vanish at the nucleus, remain
  less sensitive to these inner-core constraints.
  The observed $s$--$p$ splitting therefore reflects a differential sensitivity to
  unresolved relational structure rather than a dynamical self-interaction of the
  electron.

  At the effective level, this mechanism induces a finite upward shift of the $s$-state
  energy.
  Dimensional considerations suggest a characteristic scale of the form
  \begin{equation}
    \Delta E_{\mathrm{Lamb}}
    \sim \kappa\, \alpha^5 m_e c^2 ,
  \end{equation}
  where $\alpha$ denotes the invariant ratio between projective resolution and
  relational flux capacity, and $\kappa$ is a numerical factor of order unity encoding
  details of the local spectral structure.
  This estimate yields an energy shift of order $10^{-5}\,\mathrm{eV}$, consistent with
  the observed magnitude of the Lamb splitting.
  Crucially, the correction is intrinsically finite, reflecting the bounded nature of
  the underlying relational dynamics.

  From this viewpoint, renormalization in QED can be reinterpreted as an effective
  procedure compensating for the coarse-grained representation of unresolved spectral
  structure.
  The quantitative success of standard radiative corrections is preserved, but their
  ontological interpretation is revised.
  The Lamb shift does not arise from vacuum fluctuations, but from spectral frustration
  induced by the finite resolution and saturation properties of the projection.

  This spectral-probe logic extends naturally to other fine and hyperfine corrections,
  which similarly depend on the proximity of localized electronic states to regions of
  maximal projective compression.
  Precision spectroscopy thus provides a direct observational window into the spectral
  limits of emergent spacetime dynamics.

  \section{The Schwinger Effect as Projective Flux Saturation}
  \label{sec:schwinger-effect}

  The Schwinger effect provides a paradigmatic example of a non-perturbative phenomenon
  in quantum electrodynamics.
  It predicts the spontaneous creation of electron-positron pairs in the presence of
  a sufficiently strong electric field, with a characteristic exponential suppression
  below a critical field strength~\cite{Schwinger1951}.
  Unlike radiative corrections such as the Lamb shift, this effect cannot be captured by
  any finite-order perturbative expansion.

  In the standard interpretation, Schwinger pair production is described as a vacuum
  instability, often visualized as a tunneling process through the energy barrier
  separating negative- and positive-energy states.
  While this picture successfully reproduces the observed rate, it relies on the notion
  of a dynamically populated vacuum subject to extreme excitation.

  Within the Cosmochrony framework, the Schwinger effect admits a structural
  reinterpretation.
  Effective electromagnetic fields correspond to directed transport of relational
  relaxation flux through the non-injective projection
  $\Pi : \Omega \rightarrow \mathcal{O}$.
  This transport is subject to a finite capacity, reflecting the bounded ability of the
  projection to sustain smooth relaxation under increasing field intensity.

  At moderate field strengths, relaxation proceeds homogeneously and the projection
  remains effectively injective.
  As the imposed electric field increases, the associated relaxation flux approaches a
  saturation threshold.
  Beyond this threshold, smooth transport becomes inadmissible: additional flux cannot
  be conveyed without loss of injectivity, and the homogeneous relaxation regime becomes
  unstable.

  The onset of Schwinger pair production corresponds precisely to this saturation point.
  Rather than extracting particles from a pre-existing vacuum, the system restores
  admissibility by activating additional effective modes of the projection.
  These modes manifest as particle-antiparticle pairs, which redistribute excess
  relaxation flux into stable, projectable structures.
  Global neutrality is preserved, ensuring charge conjugation symmetry.

  From this perspective, pair production acts as a dissipation mechanism by structure
  creation.
  When the transport capacity of the effective field is exceeded, the relational
  substrate reorganizes locally, enlarging the space of admissible configurations.
  The exponential form of the Schwinger rate reflects the probabilistic breakdown of
  effective injectivity under extreme flux conditions, rather than tunneling through an
  energy barrier of the vacuum.

  This interpretation naturally aligns with Born--Infeld-type effective dynamics, in
  which field invariants are bounded and divergences are avoided through saturation.
  In Cosmochrony, such behavior is not imposed at the level of an effective Lagrangian,
  but emerges from the finite transport capacity of the relational substrate itself.

  The Schwinger effect thus exemplifies a general principle: when effective descriptions
  are driven beyond the limits of projectability, stability is restored not by divergence,
  but by reconfiguration.
  Matter creation appears as a universal relaxation channel, ensuring the continued
  consistency of the effective spacetime description under extreme conditions.

  \section{Unified Interpretation via the Fine-Structure Constant}
  \label{sec:fine-structure-constant}

  The fine-structure constant $\alpha$ occupies a singular position in modern physics.
  It governs the strength of electromagnetic interactions, controls the magnitude of
  radiative corrections in atomic systems, and sets the scale of non-perturbative
  phenomena such as the Schwinger effect.
  Despite its ubiquity and empirical precision, its origin remains unexplained within
  standard quantum electrodynamics, where it enters as a fundamental dimensionless
  input parameter.

  Within a relational pre-geometric description, $\alpha$ acquires a unified geometric
  and relational interpretation.
  Rather than characterizing a coupling between elementary fields, it emerges as an
  invariant ratio between two structural properties of the projection
  $\Pi : \Omega \rightarrow \mathcal{O}$: the resolution with which relational
  configurations can be mapped to effective observables, and the maximum flux of
  relational tension that can be transported without loss of injectivity.

  These two properties manifest differently depending on the physical regime.
  In atomic systems, where localization and spectral resolution are the dominant
  constraints, $\alpha$ controls the sensitivity of energy levels to unresolved
  relational structure.
  As shown in Section~\ref{sec:lamb-shift}, the Lamb shift reflects the response of highly
  localized states to projective spectral noise.
  The magnitude of this response is set by the ratio between the characteristic atomic
  scale and the fundamental projective resolution, encapsulated by $\alpha$.

  In contrast, in strong-field regimes such as those associated with the Schwinger
  effect, the dominant limitation is not spectral resolution but transport capacity.
  As discussed in Section~\ref{sec:schwinger-effect}, $\alpha$ governs the threshold at
  which relational flux saturates, forcing a breakdown of injectivity and the emergence
  of new effective degrees of freedom.
  The same invariant ratio that determines the scale of atomic corrections thus fixes
  the critical field strength for pair production.

  This dual role of $\alpha$ explains why phenomena as disparate as the Lamb shift and
  Schwinger pair creation are controlled by the same numerical constant within standard
  QED.
  From the projective standpoint, this is not coincidental.
  Both effects probe complementary limits of the same underlying structure: the bounded
  capacity of the projection to faithfully encode relational information into effective
  spacetime observables.

  Importantly, this interpretation does not require a modification of the numerical
  value of $\alpha$ or a departure from the quantitative predictions of QED.
  Instead, it provides an ontological explanation for the universality of $\alpha$
  across perturbative and non-perturbative regimes.
  Renormalized coupling constants and effective field strengths appear as different
  manifestations of a single structural invariant.

  The emergence of $\alpha$ as a projective invariant also clarifies its apparent
  scale-independence.
  While effective couplings may run with energy due to the structure of the effective
  description, the underlying ratio between projective resolution and flux capacity
  remains fixed.
  This explains why $\alpha$ consistently governs both low-energy precision
  spectroscopy and extreme-field phenomena without invoking additional fundamental
  parameters.

  In this unified view, the fine-structure constant is neither arbitrary nor
  mysterious.
  It encodes a balance condition between distinguishability and transport at the level
  of the relational substrate.
  The fact that this balance permeates the full range of electromagnetic phenomena
  supports the view that quantum electrodynamics operates as an effective description of
  deeper projective constraints rather than as a fundamental theory of the vacuum.

  \section{Experimental Constraints and Observational Outlook}
  \label{sec:experimental-constraints}

  Any alternative interpretation of precision quantum electrodynamics must confront
  the stringent experimental constraints imposed by modern measurements.
  In this respect, the projective framework developed in this work is not exempt from
  empirical scrutiny.
  On the contrary, the reinterpretation of vacuum-related effects as manifestations of
  bounded projectability leads to concrete and falsifiable implications.

  Precision spectroscopy of hydrogenic and hydrogen-like systems provides the most
  direct constraints on projective spectral noise.
  Measurements of the Lamb shift, hyperfine splittings, and transition frequencies in
  light atoms agree with QED predictions at the level of parts per billion.
  Within the Cosmochrony framework, this agreement implies that the effective projective
  resolution scale lies well below currently accessible atomic length scales.
  Any additional contribution arising from unresolved relational structure must therefore
  remain subdominant within the precision of existing data.

  However, the projective interpretation suggests that deviations from standard QED
  predictions, if present, should exhibit characteristic patterns.
  Rather than appearing as uniform shifts, such deviations would preferentially affect
  states with enhanced localization, higher nuclear charge, or increased sensitivity to
  short-distance structure.
  Precision comparisons between transitions with different orbital character, as well as
  systematic studies across isoelectronic sequences, offer potential avenues to isolate
  such effects.

  Strong-field phenomena provide a complementary testing ground.
  While direct observation of Schwinger pair production in static electric fields remains
  experimentally challenging, rapidly advancing laser technologies are approaching
  regimes where non-perturbative effects become accessible.
  In this context, the projective framework predicts no deviation in the existence or
  order of magnitude of the critical field, but allows for controlled modifications in
  the functional form of the pair production rate near saturation.

  In particular, departures from the standard exponential dependence may arise as the
  relational flux approaches its transport bound.
  Such deviations would manifest as changes in the pre-exponential factors or as
  environment-dependent corrections in inhomogeneous or time-dependent fields.
  These signatures differ qualitatively from those expected from conventional
  higher-order QED corrections and may therefore serve as discriminants between
  dynamical vacuum and projective saturation interpretations.

  An important aspect of the present framework is that it does not introduce additional
  free parameters at the effective level.
  The same invariant controlling atomic-scale corrections governs strong-field
  instabilities.
  As a result, constraints derived from precision spectroscopy and strong-field
  experiments are not independent, but jointly restrict the admissible structure of the
  projection.
  This interdependence enhances the falsifiability of the framework.

  Beyond laboratory experiments, the projective interpretation may also be relevant in
  astrophysical environments characterized by extreme electromagnetic fields, such as
  magnetars or relativistic plasma configurations.
  While such settings introduce additional modeling uncertainties, they offer access to
  regimes unattainable on Earth and may provide indirect constraints on saturation
  mechanisms.

  Overall, the Cosmochrony framework remains fully compatible with existing experimental
  data, while making distinctive qualitative predictions regarding the structure of
  corrections in extreme regimes.
  Future advances in precision spectroscopy, strong-field laser physics, and
  high-intensity plasma experiments will therefore play a crucial role in assessing
  whether vacuum-related phenomena are more naturally interpreted as dynamical field
  effects or as manifestations of bounded projectability in an emergent spacetime
  description.

  Beyond isolated laboratory tests, the projective framework naturally suggests a
  hierarchy of falsification across physical scales.
  Precision atomic spectroscopy probes the resolution limit of the projection through
  spectral corrections such as the Lamb shift.
  Strong-field experiments test the transport capacity of relational flux via the
  Schwinger threshold.
  If the underlying saturation mechanism is universal, analogous signatures are
  expected to arise in more extreme astrophysical and cosmological environments, where
  effective descriptions are likewise driven toward their projective limits.
  This multi-scale structure enhances the falsifiability of the framework by linking
  phenomena traditionally treated in isolation through a common set of structural
  constraints.

  \section{Discussion and Conceptual Implications}
  \label{sec:discussion}

  The reinterpretation of the Lamb shift and the Schwinger effect developed in this work
  has implications that extend beyond the specific phenomena considered.
  Together, these effects probe complementary limits of effective quantum
  electrodynamics.
  The Lamb shift explores the regime of extreme localization and spectral resolution,
  while the Schwinger effect probes the limits of flux transport under intense fields.
  Their unification within a single projective description suggests that they are not
  independent curiosities, but structural manifestations of the same underlying
  constraints.

  A central conceptual outcome of this analysis concerns the status of the quantum
  vacuum.
  In the standard formulation of QED, the vacuum is treated as a physically active
  entity, endowed with fluctuations, virtual excitations, and renormalized energies.
  While this picture is operationally successful, it leaves open deep questions about
  the ontological nature of these entities and the origin of the procedures required to
  render predictions finite.

  Within a relational pre-geometric description, vacuum-related effects acquire a
  different status.
  They do not arise from the dynamics of an underlying field-theoretic vacuum, but from
  the structural limitations of the projection mapping a pre-geometric relational
  substrate to effective spacetime observables.
  Renormalization, in this view, is not a correction to an ill-defined physical medium,
  but an effective bookkeeping device compensating for the finite resolution and
  transport capacity of the emergent description.

  This shift in perspective clarifies why quantum electrodynamics exhibits both extreme
  precision and persistent conceptual tension.
  The theory operates remarkably well within its domain of effective applicability, yet
  its interpretational difficulties signal the presence of deeper constraints not
  expressible within a purely spacetime-based formalism.
  A relational and projective description provides a setting in which these constraints
  can be made explicit, without undermining the empirical content of QED.

  An important implication of this work is that the apparent diversity of quantum
  corrections conceals a high degree of structural unity.
  Spectral shifts, non-perturbative instabilities, and coupling strengths are shown to be
  controlled by a small number of invariant projective properties.
  The fine-structure constant emerges as a central organizing parameter, encoding the
  balance between distinguishability and transport at the relational level.
  Its universality across disparate regimes ceases to be mysterious once this balance is
  recognized as fundamental.

  The results presented here reinforce the viability of a unified relational and
  projective interpretation of precision quantum electrodynamics.
  Related analyses have addressed gravitation, cosmology, and quantum non-locality
  within comparable principles.
  By embedding precision QED phenomena within the same class of structural constraints,
  the present work extends this interpretive strategy to the core of atomic and particle
  physics.
  This extension is essential for assessing whether such descriptions can provide a
  coherent effective account across a broad range of physical regimes.

  Several open questions remain.
  A more explicit characterization of the relational Laplacian spectrum and its
  coarse-grained impact on atomic dynamics would allow for quantitative refinements of
  the present interpretation.
  Similarly, a detailed modeling of flux saturation in time-dependent or inhomogeneous
  fields could sharpen predictions for strong-field experiments.
  These developments are left for future work.

  In conclusion, the Lamb shift and the Schwinger effect need not be regarded as evidence
  for a physically populated vacuum.
  They can instead be understood as signatures of bounded projectability in an emergent
  spacetime description.
  This reinterpretation preserves the empirical success of quantum electrodynamics while
  providing a coherent ontological account of its most subtle phenomena.
  As such, it supports the view that precision quantum field theories are effective
  manifestations of deeper relational structures rather than fundamental descriptions of
  physical reality.


  \section*{Funding}
    The author declares that no external funding was received for this research.

    \bmhead{Acknowledgements}
    The author acknowledges the use of large language models as a supportive tool
    for refining language, structure, and internal consistency during the
    development of this manuscript.
    All conceptual contributions, theoretical choices, and interpretations remain the sole responsibility of the author.

    \bibliography{references}

\end{document}
