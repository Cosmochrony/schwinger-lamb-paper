\section{Discussion and Conceptual Implications}
  \label{sec:discussion}

  The reinterpretation of the Lamb shift and the Schwinger effect developed in this work
  has implications that extend beyond the specific phenomena considered.
  Together, these effects probe complementary limits of effective quantum
  electrodynamics.
  The Lamb shift explores the regime of extreme localization and spectral resolution,
  while the Schwinger effect probes the limits of flux transport under intense fields.
  Their unification within a single projective framework suggests that they are not
  independent curiosities, but structural manifestations of the same underlying
  constraints.

  A central conceptual outcome of this analysis concerns the status of the quantum
  vacuum.
  In the standard formulation of QED, the vacuum is treated as a physically active
  entity, endowed with fluctuations, virtual excitations, and renormalized energies.
  While this picture is operationally successful, it leaves open deep questions about
  the ontological nature of these entities and the origin of the procedures required to
  render predictions finite.

  Within the Cosmochrony framework, vacuum-related effects acquire a different status.
  They do not arise from the dynamics of an underlying field-theoretic vacuum, but from
  the structural limitations of the projection mapping a pre-geometric relational
  substrate to effective spacetime observables.
  Renormalization, in this view, is not a correction to an ill-defined physical medium,
  but an effective bookkeeping device compensating for the finite resolution and capacity
  of the emergent description.

  This shift in perspective clarifies why quantum electrodynamics exhibits both extreme
  precision and persistent conceptual tension.
  The theory operates remarkably well within its domain of effective applicability, yet
  its interpretational difficulties signal the presence of deeper constraints not
  expressible within a purely spacetime-based formalism.
  Cosmochrony provides a framework in which these constraints are made explicit, without
  undermining the empirical content of QED.

  An important implication of this work is that the apparent diversity of quantum
  corrections conceals a high degree of structural unity.
  Spectral shifts, non-perturbative instabilities, and coupling strengths are shown to be
  controlled by a small number of invariant projective properties.
  The fine-structure constant emerges as a central organizing parameter, encoding the
  balance between distinguishability and transport at the relational level.
  Its universality across disparate regimes ceases to be mysterious once this balance is
  recognized as fundamental.

  The results presented here also reinforce the broader claim that Cosmochrony constitutes
  a unified effective framework.
  Previous works addressed gravitation, cosmology, and quantum non-locality within the
  same relational and projective principles.
  By successfully embedding precision QED phenomena into this structure, the present
  paper extends the scope of the framework to the heart of particle physics and atomic
  theory.
  This extension is essential for assessing whether Cosmochrony can serve as a coherent
  effective description across all known physical regimes.

  Several open questions remain.
  A more explicit characterization of the relational Laplacian spectrum and its
  coarse-grained impact on atomic dynamics would allow for quantitative refinements of
  the present interpretation.
  Similarly, a detailed modeling of flux saturation in time-dependent or inhomogeneous
  fields could sharpen predictions for strong-field experiments.
  These developments are left for future work.

  In conclusion, the Lamb shift and the Schwinger effect need not be regarded as evidence
  for a physically populated vacuum.
  They can instead be understood as signatures of bounded projectability in an emergent
  spacetime description.
  This reinterpretation preserves the empirical success of quantum electrodynamics while
  providing a coherent ontological account of its most subtle phenomena.
  As such, it strengthens the case that precision quantum field theories are effective
  manifestations of deeper relational structures rather than fundamental descriptions of
  physical reality.
