\section{Projective Origin of Vacuum Effects}
  \label{sec:projective-vacuum}

  In standard quantum field theory, vacuum-related effects are commonly interpreted
  as arising from fluctuations of quantized fields around a lowest-energy state.
  These fluctuations are treated as physical, albeit virtual, entities whose
  observable consequences emerge through radiative corrections and loop processes.
  While this interpretation has proven extraordinarily successful at the
  computational level, it implicitly attributes a rich dynamical structure to the
  vacuum itself.

  Within the Cosmochrony framework, no such physically populated vacuum is postulated.
  Instead, all effective physical phenomena originate from a single pre-geometric
  relational substrate~$\chi$, whose configurations are mapped to effective
  spacetime observables through a projection
  $\Pi : \Omega \rightarrow \mathcal{O}$.
  As established in earlier works, this projection is generically non-injective:
  distinct relational configurations may correspond to identical effective
  descriptions~\cite{Beau2026a}.

  Non-injectivity of the projection has direct and unavoidable physical consequences.
  In particular, it implies that the effective description cannot resolve arbitrarily
  fine relational distinctions.
  This limitation introduces a fundamental bound on localization, timing, and
  interaction resolution, independent of any dynamical fluctuations.
  Observable deviations from idealized point-like behavior therefore arise as
  structural features of the projection itself.

  We refer to this limitation as \emph{projective resolution}.
  At sufficiently low energies and moderate field intensities, the projection
  $\Pi$ behaves as effectively injective, and standard quantum and classical
  descriptions remain accurate.
  However, in regimes characterized by extreme localization or high relational
  flux, the projection saturates.
  Beyond this regime, additional relational information cannot be faithfully
  transported into the effective spacetime description.

  The saturation of projective resolution manifests observationally as corrections
  to otherwise degenerate or symmetric configurations.
  Crucially, these corrections do not signal the presence of additional degrees of
  freedom, but rather the breakdown of perfect correspondence between relational
  configurations and effective observables.
  In this sense, so-called vacuum effects are reinterpreted as \emph{projection
artifacts}, encoding the finite capacity and resolution of the mapping from
  $\chi$ to spacetime.

  This perspective provides a unified reinterpretation of phenomena traditionally
  treated as conceptually distinct.
  Corrections attributed to vacuum polarization, self-energy, or zero-point
  fluctuations are recast as manifestations of bounded projectability.
  They reflect the fact that effective fields and particles are not fundamental
  entities, but emergent descriptions constrained by the structural properties of
  the underlying relational substrate.

  In the following sections, this projective interpretation is applied to two
  paradigmatic quantum electrodynamical phenomena.
  The Lamb shift is shown to arise from the sensitivity of highly localized atomic
  states to projective spectral noise.
  The Schwinger effect is interpreted as a saturation-induced instability, in which
  excess relational flux forces a local reconfiguration of the substrate.
  Together, these effects illustrate how precision QED phenomena emerge naturally
  from the same projective principles governing gravitation and cosmology within
  the Cosmochrony framework.
