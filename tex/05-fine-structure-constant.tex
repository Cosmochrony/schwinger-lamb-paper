\section{Unified Interpretation via the Fine-Structure Constant}
  \label{sec:fine-structure-constant}

  The fine-structure constant $\alpha$ occupies a singular position in modern physics.
  It governs the strength of electromagnetic interactions, controls the magnitude of
  radiative corrections in atomic systems, and sets the scale of non-perturbative
  phenomena such as the Schwinger effect.
  Despite its ubiquity and empirical precision, its origin remains unexplained within
  standard quantum electrodynamics, where it enters as a fundamental dimensionless
  input parameter.

  Within a relational pre-geometric description, $\alpha$ acquires a unified geometric
  and relational interpretation.
  Rather than characterizing a coupling between elementary fields, it emerges as an
  invariant ratio between two structural properties of the projection
  $\Pi : \Omega \rightarrow \mathcal{O}$: the resolution with which relational
  configurations can be mapped to effective observables, and the maximum flux of
  relational tension that can be transported without loss of injectivity.

  These two properties manifest differently depending on the physical regime.
  In atomic systems, where localization and spectral resolution are the dominant
  constraints, $\alpha$ controls the sensitivity of energy levels to unresolved
  relational structure.
  As shown in Section~\ref{sec:lamb-shift}, the Lamb shift reflects the response of highly
  localized states to projective spectral noise.
  The magnitude of this response is set by the ratio between the characteristic atomic
  scale and the fundamental projective resolution, encapsulated by $\alpha$.

  In contrast, in strong-field regimes such as those associated with the Schwinger
  effect, the dominant limitation is not spectral resolution but transport capacity.
  As discussed in Section~\ref{sec:schwinger-effect}, $\alpha$ governs the threshold at
  which relational flux saturates, forcing a breakdown of injectivity and the emergence
  of new effective degrees of freedom.
  The same invariant ratio that determines the scale of atomic corrections thus fixes
  the critical field strength for pair production.

  This dual role of $\alpha$ explains why phenomena as disparate as the Lamb shift and
  Schwinger pair creation are controlled by the same numerical constant within standard
  QED.
  From the projective standpoint, this is not coincidental.
  Both effects probe complementary limits of the same underlying structure: the bounded
  capacity of the projection to faithfully encode relational information into effective
  spacetime observables.

  Importantly, this interpretation does not require a modification of the numerical
  value of $\alpha$ or a departure from the quantitative predictions of QED.
  Instead, it provides an ontological explanation for the universality of $\alpha$
  across perturbative and non-perturbative regimes.
  Renormalized coupling constants and effective field strengths appear as different
  manifestations of a single structural invariant.

  The emergence of $\alpha$ as a projective invariant also clarifies its apparent
  scale-independence.
  While effective couplings may run with energy due to the structure of the effective
  description, the underlying ratio between projective resolution and flux capacity
  remains fixed.
  This explains why $\alpha$ consistently governs both low-energy precision
  spectroscopy and extreme-field phenomena without invoking additional fundamental
  parameters.

  In this unified view, the fine-structure constant is neither arbitrary nor
  mysterious.
  It encodes a balance condition between distinguishability and transport at the level
  of the relational substrate.
  The fact that this balance permeates the full range of electromagnetic phenomena
  supports the view that quantum electrodynamics operates as an effective description of
  deeper projective constraints rather than as a fundamental theory of the vacuum.
