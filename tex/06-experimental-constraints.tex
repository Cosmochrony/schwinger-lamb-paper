\section{Experimental Constraints and Observational Outlook}
  \label{sec:experimental-constraints}

  Any alternative interpretation of precision quantum electrodynamics must confront
  the stringent experimental constraints imposed by modern measurements.
  In this respect, the projective framework developed in this work is not exempt from
  empirical scrutiny.
  On the contrary, the reinterpretation of vacuum-related effects as manifestations of
  bounded projectability leads to concrete and falsifiable implications.

  Precision spectroscopy of hydrogenic and hydrogen-like systems provides the most
  direct constraints on projective spectral noise.
  Measurements of the Lamb shift, hyperfine splittings, and transition frequencies in
  light atoms agree with QED predictions at the level of parts per billion.
  Within the Cosmochrony framework, this agreement implies that the effective projective
  resolution scale lies well below currently accessible atomic length scales.
  Any additional contribution arising from unresolved relational structure must therefore
  remain subdominant within the precision of existing data.

  However, the projective interpretation suggests that deviations from standard QED
  predictions, if present, should exhibit characteristic patterns.
  Rather than appearing as uniform shifts, such deviations would preferentially affect
  states with enhanced localization, higher nuclear charge, or increased sensitivity to
  short-distance structure.
  Precision comparisons between transitions with different orbital character, as well as
  systematic studies across isoelectronic sequences, offer potential avenues to isolate
  such effects.

  Strong-field phenomena provide a complementary testing ground.
  While direct observation of Schwinger pair production in static electric fields remains
  experimentally challenging, rapidly advancing laser technologies are approaching
  regimes where non-perturbative effects become accessible.
  In this context, the projective framework predicts no deviation in the existence or
  order of magnitude of the critical field, but allows for controlled modifications in
  the functional form of the pair production rate near saturation.

  In particular, departures from the standard exponential dependence may arise as the
  relational flux approaches its transport bound.
  Such deviations would manifest as changes in the pre-exponential factors or as
  environment-dependent corrections in inhomogeneous or time-dependent fields.
  These signatures differ qualitatively from those expected from conventional
  higher-order QED corrections and may therefore serve as discriminants between
  dynamical vacuum and projective saturation interpretations.

  An important aspect of the present framework is that it does not introduce additional
  free parameters at the effective level.
  The same invariant controlling atomic-scale corrections governs strong-field
  instabilities.
  As a result, constraints derived from precision spectroscopy and strong-field
  experiments are not independent, but jointly restrict the admissible structure of the
  projection.
  This interdependence enhances the falsifiability of the framework.

  Beyond laboratory experiments, the projective interpretation may also be relevant in
  astrophysical environments characterized by extreme electromagnetic fields, such as
  magnetars or relativistic plasma configurations.
  While such settings introduce additional modeling uncertainties, they offer access to
  regimes unattainable on Earth and may provide indirect constraints on saturation
  mechanisms.

  Overall, the Cosmochrony framework remains fully compatible with existing experimental
  data, while making distinctive qualitative predictions regarding the structure of
  corrections in extreme regimes.
  Future advances in precision spectroscopy, strong-field laser physics, and
  high-intensity plasma experiments will therefore play a crucial role in assessing
  whether vacuum-related phenomena are more naturally interpreted as dynamical field
  effects or as manifestations of bounded projectability in an emergent spacetime
  description.

  Beyond isolated laboratory tests, the projective framework naturally suggests a
  hierarchy of falsification across physical scales.
  Precision atomic spectroscopy probes the resolution limit of the projection through
  spectral corrections such as the Lamb shift.
  Strong-field experiments test the transport capacity of relational flux via the
  Schwinger threshold.
  If the underlying saturation mechanism is universal, analogous signatures are
  expected to arise in more extreme astrophysical and cosmological environments, where
  effective descriptions are likewise driven toward their projective limits.
  This multi-scale structure enhances the falsifiability of the framework by linking
  phenomena traditionally treated in isolation through a common set of structural
  constraints.
