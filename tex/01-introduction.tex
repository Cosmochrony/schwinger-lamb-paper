\section{Introduction and Motivation}
  \label{sec:introduction}

  Quantum Electrodynamics (QED) stands as one of the most accurate physical theories
  ever constructed, achieving extraordinary agreement between theoretical predictions
  and experimental measurements across a wide range of phenomena~\cite{Weinberg1995}.
  Among its most emblematic successes are the Lamb shift in hydrogenic atoms and the
  non-perturbative prediction of electron-positron pair production in ultra-strong
  electric fields, known as the Schwinger effect~\cite{Lamb1947,Schwinger1951}.

  Despite their empirical precision, these phenomena raise persistent conceptual
  questions regarding the ontological status of the quantum vacuum.
  In the standard formulation, both effects are attributed to vacuum fluctuations
  and radiative corrections arising from the interaction of charged particles with
  quantized electromagnetic fields.
  The vacuum is thus treated as a dynamical entity, endowed with structure, energy,
  and virtual excitations, whose effects must be regularized and renormalized to
  yield finite physical predictions~\cite{PeskinSchroeder1995}.

  Within the Cosmochrony framework, the quantum vacuum is not regarded as a fundamental
  physical medium.
  Instead, all effective physical observables arise from a non-injective projection
  $\Pi$ of a single pre-geometric relational substrate~$\chi$ onto an effective
  spacetime description.
  As developed in previous works, the non-injectivity of this projection implies that
  distinct relational configurations may correspond to identical effective observables,
  leading to intrinsic limits in localization, resolution, and transport~\cite{Beau2026a}.

  From this perspective, phenomena traditionally attributed to vacuum fluctuations
  may be reinterpreted as manifestations of projective limitations rather than
  dynamical excitations of an underlying field.
  In regimes of high informational density or extreme field intensity, the projection
  $\Pi$ ceases to be effectively injective, and observable corrections emerge as
  structural effects associated with bounded resolution and saturation of relational
  flux.

  The present work applies this projective interpretation to two cornerstone effects
  of QED: the Lamb shift and the Schwinger pair production mechanism.
  We argue that the Lamb shift can be understood as a form of projective spectral noise,
  arising from the finite resolvability of interactions localized at atomic scales.
  In this view, the energy level shifts of $s$-states relative to $p$-states do not
  reflect electron self-energy in a fluctuating vacuum, but the sensitivity of highly
  localized states to the spectral structure of the underlying relational Laplacian.

  Similarly, the Schwinger effect is reinterpreted as a saturation phenomenon associated
  with a bounded capacity of the relational substrate to transport effective
  electromagnetic flux.
  When this capacity is exceeded, the projection becomes unstable, and the system
  relaxes through a local re-stratification of the substrate, manifesting as the
  creation of particle-antiparticle pairs.
  Pair production is therefore not viewed as extraction from the vacuum, but as an
  inevitable consequence of non-injective projection under extreme field conditions.

  A central outcome of this unified interpretation is a revised understanding of the
  fine-structure constant~$\alpha$.
  Rather than appearing as an unexplained fundamental parameter, $\alpha$ emerges as
  an invariant ratio between projective resolution and relational flux capacity,
  governing both the magnitude of atomic-level corrections and the threshold behavior
  of strong-field phenomena.

  The goal of this paper is not to challenge the quantitative success of QED, but to
  provide an alternative ontological interpretation of its most subtle predictions.
  By reframing the Lamb shift and the Schwinger effect as projective and saturation
  phenomena, respectively, this work aims to demonstrate that precision quantum
  electrodynamics can be consistently embedded within a pre-geometric relational
  framework.
  This embedding offers a coherent account of vacuum-related effects without invoking
  a physically populated vacuum, and strengthens the claim that Cosmochrony provides a
  unified effective description spanning gravitation, quantum non-locality, and
  electromagnetic interactions.
